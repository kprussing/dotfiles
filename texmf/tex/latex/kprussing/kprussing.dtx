% \iffalse meta-comment
%
% Copyright (c) 2019 by Keith F. Prussing <kprussing74@gmail.com>
%
% This work may be distributed and/or modified under the conditions of
% the LaTeX Project Public License, either version 1.3c of this license
% or (at your option) any later version.  The latest version of this
% license is in
%
%     http://www.latex-project.org/lppl.txt
%
% and version 1.3c or later is part of all distributions of LaTeX
% version 2005/12/01 or later.
%
% This work has the LPPL maintenance status `maintained'.
%
% The Current Maintainer of this work is Keith F. Prussing.
%
% \fi
%
% \iffalse
%<*driver>
\ProvidesFile{\jobname.dtx}
%</driver>
%<common>\NeedsTeXFormat{LaTeX2e}[2003/12/01]
%<report>\ProvidesPackage{kprussing}
%<fonts>\ProvidesPackage{kprussing-font}
%<presentation>\ProvidesPackage{beamerthemekprussing}
%<beamercolor>\ProvidesPackage{beamercolorthemekprussing}
%<beamerfont>\ProvidesPackage{beamerfontthemekprussing}
%<beamerinner>\ProvidesPackage{beamerinnerthemekprussing}
%<beamerouter>\ProvidesPackage{beamerouterthemekprussing}
%<*common>
    [2019/01/17 v0.1 kprussing personal themes]
%</common>
%
%<*driver>
\documentclass{ltxdoc}
\EnableCrossrefs{}
\CodelineIndex{}
\RecordChanges{}
\usepackage{hologo}
\usepackage{kprussing}
\usepackage[columns=1]{idxlayout}
\begin{document}
    \DocInput{\jobname.dtx}
\end{document}
%</driver>
% \fi
%
% \CheckSum{0}
%
% \CharacterTable
%  {Upper-case \A\B\C\D\E\F\G\H\I\J\K\L\M\N\O\P\Q\R\S\T\U\V\W\X\Y\Z
%   Lower-case \a\b\c\d\e\f\g\h\i\j\k\l\m\n\o\p\q\r\s\t\u\v\w\x\y\z
%   Digits     \0\1\2\3\4\5\6\7\8\9
%   Exclamation   \!     Double quote  \"     Hash (number) \#
%   Dollar        \$     Percent       \%     Ampersand     \&
%   Acute accent  \'     Left paren    \(     Right paren   \)
%   Asterisk      \*     Plus          \+     Comma         \,
%   Minus         \-     Point         \.     Solidus       \/
%   Colon         \:     Semicolon     \;     Less than     \<
%   Equals        \=     Greater than  \>     Question mark \?
%   Commercial at \@     Left bracket  \[     Backslash     \\
%   Right bracket \]     Circumflex    \^     Underscore    \_
%   Grave accent  \`     Left brace    \{     Vertical bar  \|
%   Right brace   \}     Tilde         \~}
%
% \changes{v0.1}{2019/01/17}{Initial version}
% \GetFileInfo{\jobname.dtx}
%
%^^A Package macros
% \DoNotIndex{\CurrentOption, \DeclareOption, \ExecuteOptions}
% \DoNotIndex{\PackageError, \PackageWarning, \PassOptionsToPackage}
% \DoNotIndex{\ProcessOptions, \RequirePackage}
%^^A Font macros
% \DoNotIndex{\bfseries, \familydefault, \Huge, \Large, \sfseries}
% \DoNotIndex{\normalsize, \scriptsize, \setmainfont, \setmathfont}
% \DoNotIndex{\setmonofont, \setsansfont, \sfdefault, \textsc, \textsf}
% \DoNotIndex{\texttt}
%^^A Low level macros
% \DoNotIndex{\begin, \bigskip, \centering, \else, \end, \fi, \hfill}
% \DoNotIndex{\hspace, \ifPDFTeX, \newcommand, \newif, \null}
% \DoNotIndex{\paperheight, \paperwidth, \par, \providecommand}
% \DoNotIndex{\relax, \renewcommand, \textbullet, \textcopyright}
% \DoNotIndex{\textwidth, \vfill, \vspace}
%^^A Beamer macros
% \DoNotIndex{\DeclareOptionBeamer, \ExecuteOptionsBeamer}
% \DoNotIndex{\defbeamertemplate, \insertauthor, \insertbiblabel}
% \DoNotIndex{\insertdate, \insertinstitute, \insertpart}
% \DoNotIndex{\insertpartnumber, \insertshortauthor, \insertshortdate}
% \DoNotIndex{\insertsubenumlabel, \insertsubsubenumlabel}
% \DoNotIndex{\insertsubtitle, \inserttitle, \mode}
% \DoNotIndex{\ProcessOptionsBeamer, \setbeamercolor, \setbeamerfont}
% \DoNotIndex{\setbeamertemplate, \usebeamercolor, \usebeamerfont}
% \DoNotIndex{\usebeamertemplate, \usecolortheme, \usefonttheme}
% \DoNotIndex{\useinnertheme, \useoutertheme, \usetheme}
%^^A Miscellaneous package macros
% \DoNotIndex{\definecolor, \hypersetup, \lstinline, \texorpdfstring}
%^^A Package conditionals
% \DoNotIndex{\kprussing@dejavufalse, \kprussing@dejavutrue}
% \DoNotIndex{\kprussing@sourceprofalse, \kprussing@sourceprotrue}
% \DoNotIndex{\kprussing@unicodemathfalse, \kprussing@unicodemathtrue}
%
% \title{Keith's \LaTeX{} styles \thanks{This document
% corresponds to \textsf{kprussing}~\fileversion, dated~\filedate.}}
%
% \author{Keith F. Prussing \\ \texttt{kprussing74@gmail.com}}
%
% \maketitle
%
% \begin{abstract}
% This package is an attempt to provide a set of themes for \LaTeX{}
% that will collect my preferred settings into a set of styles and
% themes.  It provides a beamer theme for presentations and a style for
% use with the |report| or |book| |documentclass| for formatting
% reports.  The aim of this side project it so provide an uniform set of
% tools for generating nicely type materials in a consistent style.
% \end{abstract}
%
% \section{Introduction}
%
% This is a collection of \LaTeX{} macros and settings to create nicely
% typeset materials in my preferred styles.  It provides a beamer theme
% and macros for formatting a report.  The beamer theme defines the
% outer, inner, color, and font themes.  They can be used individually,
% but will most often be used together.  In addition to the two main
% themes, this provides common definitions for my preferred fonts and
% colors.
%
% \section{Usage}
%
% To install the package, simply run the provided |kprussing.ins| file
% through your preferred \hologo{pdfLaTeX}.  Then, copy the resultant
% |.sty| and |.def| files to a location where \LaTeX{} can find them.
% To generate the documentation, run this file (|kprussing.dtx|) through
% \hologo{pdfLaTeX} twice.  The second time is only to make sure the
% references are typeset correctly.  Using the theme or package is
% simply a matter of adding |\usetheme{kprussing}| or
% |\usepackage{kprussing}\| to your preamble.
%
% The better way to build the files is to simply use \textsf{SCons} with
% the provided |SConstruct|.
%
% \subsection{General stylistic decisions}
%
% This section details the general design decisions that apply to both
% the report package and the presentation theme.  These are the
% appropriate fonts and colors.  Beyond that, we provide a set of
% utility macros for typesetting software component commonly encountered
% in documentation and presentations.
%
% \subsubsection{Fonts and colors}
%
% After using for a while at work, I find I really like the Source Pro
% family of fonts.  They have good coverage of the correct Unicode
% points and have (most) of the correct styles.  The DejaVu family is a
% close second.  In fact, we mix the two in a limited sense to fill out
% some missing styles.  Both of these fonts are OpenType and ship with
% most modern \LaTeX{} distributions.  The correct fonts are loaded
% conditionally based on if \hologo{LuaLaTeX}, \hologo{XeLaTeX}, or
% \hologo{pdfLaTeX}.
%
% Source Sans Pro offers essentially all of the font features we desire.
% The serif font is missing italic and small caps, but the italic font
% was recently (August 2018) released in version 2.007.  So, the
% italics are simply a matter of waiting for the main distributions to
% updated.  With \textsf{fontspec} active in \hologo{LuaLaTeX} or
% \hologo{XeLaTeX}, we can work around these deficiencies by assigning a
% substitute font.  For small caps, we fall back to the sans-serif font.
% It exists and is generally well matched to the serif, and it provides
% an extra level of distinction from its neighbors which is a desirable
% effect.  For the italics, we can temporarily use the DejaVu Serif
% which is passably well matched.  This is a temporary work around while
% we wait for the true italic font.
%
% \subsubsection{Stylistic macros}
%
% \DescribeMacro{\software}
% \DescribeMacro{\variable}
% \DescribeMacro{\program}
% \DescribeMacro{\subprogram}
% When presenting my work, I often need to describe software at
% multiple levels.  Specifically, I need to speak about ``software,''
% ``variables,'' ``programs,'' and ``subprograms.''  To ensure
% consistency, this package provides the four semantic markup macros
% |\software|, |\variable|, |\program|, and |\subprogram|.  Each takes a
% mandatory argument and typesets it accordingly.  The notation for the
% subprograms derives from the Fortran notation of subroutines and
% functions contained within another programming unit.
%
% \subsection{Presentation specific decisions}
% \subsubsection{Color theme}
% \subsubsection{Font theme}
% \subsubsection{Inner theme}
% \subsubsection{Outer theme}
% \subsubsection{Title page}
%
% At the moment, it is very simple.  It has the information centered and
% is quite boring.
%
% \StopEventually{
% \PrintChanges{}
% \PrintIndex{}
%}
%
% \section{Implementation}
%
% \subsection{General stylistic decisions}
% \changes{v0.1}{2019/01/17}{Define my main style}
%
% The main style file simply loads the appropriate files.
%
% \iffalse
%<*report>
% \fi
%    \begin{macrocode}
\DeclareOption{sourcepro}{\PassOptionsToPackage{sourcepro}%
                                               {kprussing-font}}
\DeclareOption{dejavu}{\PassOptionsToPackage{dejavu}%
                                            {kprussing-font}}
\ProcessOptions\relax
\RequirePackage{kprussing-font}
%    \end{macrocode}
% \iffalse
%</report>
% \fi
%
% \subsubsection{Fonts}
% \changes{v0.1}{2019/01/17}{Define my preferred fonts}
%
% The Source Pro family of fonts has good coverage of the Unicode points
% I frequently use.  It also has a matched serif, sans-serif, and
% monospaced version that maintains a consistent style.  The DejaVu
% family is a good alternate.  To handle toggling options, we need some
% logical flags
%
% \iffalse
%<*fonts>
% \fi
%
% To handle toggling options, we need some logicals.  We want flags for
% the Source Pro family and the DejaVu family of fonts.  We also declare
% a flag for Unicode math that will only be used with \hologo{LuaLaTeX}
% and \hologo{XeLaTeX}.
%    \begin{macrocode}
\RequirePackage{iftex}
\newif\ifkprussing@sourcepro
\newif\ifkprussing@dejavu
\newif\ifkprussing@unicodemath
%    \end{macrocode}
%
% Now we can declare the options to toggle the options.
%    \begin{macrocode}
\DeclareOption{sourcepro}{\kprussing@sourceprotrue
                          \kprussing@dejavufalse}
\DeclareOption{dejavu}{\kprussing@sourceprofalse
                       \kprussing@dejavutrue}
\DeclareOption{math}{%
    \ifPDFTeX{}
        \PackageError{kprussing-font}%
                     {Unicode math is not supported with PDFLaTeX}
    \else
        \PackageWarning{kprussing-font}%
                       {Unicode math support is still experimental}
        \kprussing@unicodemathtrue
    \fi
}
%    \end{macrocode}
%
% And set the Source Pro family as the default.
%    \begin{macrocode}
\ExecuteOptions{sourcepro}
\ProcessOptions\relax
%    \end{macrocode}
%
% Now we need to load the fonts.  We start with the Source Pro family of
% fonts.
%    \begin{macrocode}
\ifkprussing@sourcepro
    \ifPDFTeX{}
        \RequirePackage{sourceserifpro}
        \RequirePackage{sourcesanspro}
        \RequirePackage{sourcecodepro}
        \RequirePackage[T1]{fontenc}
        \RequirePackage[utf8]{inputenc}
    \else
        \RequirePackage{fontspec}
%    \end{macrocode}
% The serif font does not have small caps or italic shapes yet.  For
% now, we use the sans-serif small caps and the DejaVu Serif italics.
% The italics will need to change when the updated font is released.
%    \begin{macrocode}
        \setmainfont{Source Serif Pro}[%
            SmallCapsFont={Source Sans Pro},
            SmallCapsFeatures={%
                Letters=SmallCaps,
                Scale=MatchUppercase
            },
            ItalicFont=DejaVuSerif-Italic,
            BoldItalicFont=DejaVuSerif-BoldItalic,
            SlantedFont=DejaVuSerif-Italic,
            BoldSlantedFont=DejaVuSerif-BoldItalic
        ]
        \setsansfont{Source Sans Pro}
        \setmonofont{Source Code Pro}
    \fi
\fi
%    \end{macrocode}
%
% Next, we load the DejaVu family of fonts.  We don't have a good small
% capitals in the DejaVu family so we fall back on the Source Sans Pro.
% This mirrors the decision above to use DejaVu italics for the Source
% Serif Pro.
%    \begin{macrocode}
\ifkprussing@dejavu
    \ifPDFTeX{}
        \RequirePackage{dejavu}
        \RequirePackage[T1]{fontenc}
        \RequirePackage[utf8]{inputenc}
    \else
        \RequirePackage{fontspec}
        \setmainfont{DejaVu Serif}[%
            SmallCapsFont={Source Sans Pro},
            SmallCapsFeatures={%
                Letters=SmallCaps,
                Scale=MatchUppercase
            }
        ]
        \setsansfont{DejaVu Sans}[%
            SmallCapsFont={Source Sans Pro},
            SmallCapsFeatures={%
                Letters=SmallCaps,
                Scale=MatchUppercase
            }
        ]
        \setmonofont{DejaVu Sans Mono}
    \fi
\fi
%    \end{macrocode}
%
% Unicode math is finicky.  So, we define it at a per document level for
% now while we figure it out.  The problem is the math fonts \emph{must}
% be loaded before the unicode-math.  To make sure we have a way to get
% \emph{something}, we offer a package option.
%    \begin{macrocode}
\ifPDFTeX{}
\else
    \ifkprussing@unicodemath
        \RequirePackage{amsmath}
        \RequirePackage{amsfonts}
        \RequirePackage{unicode-math}
        \setmathfont{TeX Gyre Pagella Math}
        \RequirePackage{lualatex-math}
    \fi
\fi
%    \end{macrocode}
% \iffalse
%</fonts>
% \fi
%
% \subsubsection{Stylistic macros}
% \changes{v0.1}{2019/01/17}{Define syntactic markup macros}
%
% \iffalse
%<*report>
% \fi
% \begin{macro}{\software}
% Use small caps for software.  This provides a distinctive appearance
% from the main text similar to the old style of using all uppercase.
% But the small caps do not disrupt the visual flow of the sentence as
% much.
%    \begin{macrocode}
\newcommand{\software}[1]{%
    \textsc{#1}
}
%    \end{macrocode}
% \end{macro}
%
% The |\program|, |\subprogram|, and |\variable| macros are very closely
% related because they all describe components within a larger piece of
% software.  The natural choice would seem to be a monospace font or a
% verbatim region; however, we can take advantage of the
% \textsf{listings} package to offer highlighting if the user has
% configured this.
%
% \begin{macro}{\program}
% We start by defining the |\program| to safely use either the inline
% listing or fall back to the monospace font in case it needs to be the
% target of a hyperlink.
%    \begin{macrocode}
\RequirePackage{listings}
\newcommand{\program}[1]{%
    \texorpdfstring{\lstinline{#1}}{\texttt{#1}}
}
%    \end{macrocode}
% \end{macro}
% \begin{macro}{\subprogram}
% The subprogram simply calls the |\program| because they are so closely
% related.
%    \begin{macrocode}
\newcommand{\subprogram}[1]{\program{#1}}
%    \end{macrocode}
% \end{macro}
% \begin{macro}{\variable}
% As with the |\subprogram|, we simply use the |\program| macro to get
% the common typesetting.
%    \begin{macrocode}
\newcommand{\variable}[1]{\program{#1}}
%    \end{macrocode}
% \end{macro}
%
% \iffalse
%</report>
% \fi
%
% \subsubsection{Main presentation theme}
% \changes{v0.1}{2019/01/17}{Collect the presentation themes together}
%
% \iffalse
%<*presentation>
% \fi
% Collect all of the themes into one global beamer theme.  First, we
% provide an option to define the color scheme that is simply passed to
% the color theme.
%    \begin{macrocode}
% Color options
\DeclareOptionBeamer{light}{%
    \PassOptionsToPackage{light}{beamercolorthemekprussing}
}
\ExecuteOptionsBeamer{light}
\ProcessOptionsBeamer\relax
%    \end{macrocode}
% Now we simply load all of the themes along with the default style.
%    \begin{macrocode}
\RequirePackage{kprussing}
\mode<presentation>
\usecolortheme{kprussing}
\usefonttheme{kprussing}
\useoutertheme{kprussing}
\useinnertheme{kprussing}
\mode<all>
%    \end{macrocode}
% \iffalse
%</presentation>
% \fi
%
% \subsubsection{Presentation color theme}
% \changes{v0.1}{2019/01/17}{Set the color scheme and the light version}
%
% \iffalse
%<*beamercolor>
% \fi
% First, define a dark text on a light background color palette.  The
% color palette will be used to set the major components of the
% presentation to give it a consistent feel.
%    \begin{macrocode}
\DeclareOptionBeamer{light}{%
    \definecolor{palette-primary}   {RGB}{  0, 37, 76}
    \definecolor{palette-secondary} {RGB}{128,128,128}
    \definecolor{palette-tertiary}  {RGB}{188,161,110}
    \definecolor{palette-quaternary}{RGB}{ 34, 42, 53}
    \definecolor{main-text}         {RGB}{ 34, 42, 53}
    \definecolor{background-text}   {RGB}{173,185,202}
}
\ExecuteOptionsBeamer{light}
%    \end{macrocode}
% And process the options.
%    \begin{macrocode}
\DeclareOptionBeamer*{\PackageWarning{beamercolorthemekprussing}%
                      {Unknown `\CurrentOption'}}
\ProcessOptionsBeamer\relax
%    \end{macrocode}
%
% Now we are ready to define all the colors for the palette.  We will
% focus mainly on defining the style and tidying up odds and ends.
% First, the very basic color scheme.
%    \begin{macrocode}
\mode<presentation>
\setbeamercolor*{normal text}       {fg=main-text,
                                     bg=background-text!10!white}
\setbeamercolor*{structure}         {fg=palette-primary}
\setbeamercolor*{palette primary}   {fg=palette-primary}
\setbeamercolor*{palette secondary} {fg=palette-secondary}
\setbeamercolor*{palette tertiary}  {fg=palette-tertiary}
\setbeamercolor*{palette quaternary}{fg=palette-quaternary}
%    \end{macrocode}
% Next explicit call outs
%    \begin{macrocode}
\setbeamercolor*{alerted text}      {use=palette tertiary}
\setbeamercolor*{example text}      {use=palette secondary}
%    \end{macrocode}
% And use the primary palette for titles
%    \begin{macrocode}
\setbeamercolor*{titlelike}         {use=palette primary}
\setbeamercolor*{frametitle}        {use=palette primary}
%    \end{macrocode}
% And define the headline color using a reverse color scheme
%    \begin{macrocode}
\setbeamercolor*{headline}          {fg=palette tertiary.fg,
                                     bg=structure.fg}
%    \end{macrocode}
% Place the title slide colors here and not with the title slide to
% allow a user to override these choices in a single location.
%    \begin{macrocode}
\setbeamercolor*{title}             {parent=palette primary}
\setbeamercolor*{subtitle}          {parent=palette primary}
\setbeamercolor*{author}            {parent=palette primary}
\setbeamercolor*{institute}         {parent=palette tertiary}
\setbeamercolor*{date}              {parent=palette primary}
%    \end{macrocode}
% And establish the \textsf{hyperef} colors.  Beamer already takes care
% of loading \textsf{hyperref} so we know it is available.
%    \begin{macrocode}
\hypersetup{colorlinks=true,
            linkcolor=normal text.fg,
            urlcolor=palette tertiary.fg,
            citecolor=palette tertiary.fg,
            anchorcolor=normal text.fg}
\mode<all>
%    \end{macrocode}
% \iffalse
%</beamercolor>
% \fi
%
% \subsubsection{Presentation font theme}
% \changes{v1.0}{2019/01/17}{Set the presentation fonts}
%
% \iffalse
%<*beamerfont>
% \fi
% We know we want to use a proper font designed by professionals, but do
% we want Source Pro or DejaVu?  This means we need options.
%    \begin{macrocode}
\DeclareOptionBeamer{sourcepro}{%
    \PassOptionsToPackage{sourcepro}{kprussing-font}
}
\DeclareOptionBeamer{dejavu}{%
    \PassOptionsToPackage{dejavu}{kprussing-font}
}
\DeclareOptionBeamer{math}{%
    \PassOptionsToPackage{math}{kprussing-font}
}
\ExecuteOptionsBeamer{sourcepro}
\DeclareOptionBeamer*{\PackageWarning{beamerfontthemekprussing}%
                       {Unknown `\CurrentOption'}}
\ProcessOptionsBeamer\relax
%    \end{macrocode}
%
% Next, we just load the fonts
%    \begin{macrocode}
\usefonttheme{professionalfonts}
\RequirePackage{kprussing-font}
%    \end{macrocode}
%
% Now we just need to use the sans-serif font and adjust the fonts.
%    \begin{macrocode}
\mode<presentation>
\renewcommand{\familydefault}{\sfdefault}
\setbeamerfont*{frametitle}  {size=\Large, series=\bfseries}
\setbeamerfont*{normal text} {size=\normalsize}
\setbeamerfont*{title}       {size=\Huge, series=\bfseries}
\setbeamerfont*{subtitle}    {size=\Large, series=\bfseries}
\setbeamerfont*{author}      {size=\normalsize}
\setbeamerfont*{institute}   {size=\scriptsize}
\setbeamerfont*{date}        {size=\normalsize}
\mode<all>
%    \end{macrocode}
% \iffalse
%</beamerfont>
% \fi
%
% \subsubsection{Presentation inner theme}
% \changes{v0.1}{2019/01/17}{Add inner theme (with simple title)}
%
% \iffalse
%<*beamerinner>
% \fi
% The inner theme defines the title slide, the parts page, a macro to
% place the image on the outline slide, fixes the item bullets, and
% fixes the bibliography.
%
% Now we define the title page.
%    \begin{macrocode}
\setbeamertemplate{title page}
{%
    \centering
    \vfill
    \usebeamercolor[fg]{title}
    \usebeamerfont{title}\inserttitle\par
    \usebeamercolor[fg]{subtitle}
    \usebeamerfont{subtitle}\insertsubtitle\par
    \bigskip
    \usebeamercolor[fg]{author}
    \usebeamerfont{author}\insertauthor\par
    \usebeamercolor[fg]{institute}
    \usebeamerfont{institute}\insertinstitute\par
    \bigskip
    \usebeamercolor[fg]{date}
    \usebeamerfont{date}\insertdate\par
}
%    \end{macrocode}
%
% The part page uses the same template as the title page with the part
% number and name in the place of the affiliation.
%    \begin{macrocode}
\setbeamertemplate{part page}
{%
    \centering
    \vfill
    \usebeamercolor[fg]{title}
    \usebeamerfont{title}\inserttitle\par
    \usebeamercolor[fg]{subtitle}
    \usebeamerfont{subtitle}\insertsubtitle\par
    \bigskip
    \usebeamercolor[fg]{title}
    \usebeamercolor[fg]{title}Part \insertpartnumber: \insertpart%
    \par
    \bigskip
    \usebeamercolor[fg]{author}
    \usebeamerfont{author}\insertauthor\par
    \usebeamercolor[fg]{date}
    \usebeamerfont{date}\insertdate\par
}
%    \end{macrocode}
%
% Now for the small details.  Use an actual bullet for the itemize list
% and strip the extra levels from the enumerated lists.
%    \begin{macrocode}
\setbeamertemplate{itemize items}{\bfseries{\textbullet}}
\setbeamertemplate{enumerate subitem}{\insertsubenumlabel}
\setbeamertemplate{enumerate subsubitem}{\insertsubsubenumlabel}
%    \end{macrocode}
%
% Force the bibliography to not be hideous.  If you have more than one
% page of references, you want to tell beamer to put the continuation
% section label from the second slide onwards
%    \begin{macrocode}
\setbeamertemplate{frametitle continuation}[from second]
%    \end{macrocode}
% Now get rid of all the colors and make the font a single (smaller)
% size.
%    \begin{macrocode}
\setbeamerfont{bibliography entry title}{size=\scriptsize}
\setbeamercolor*{bibliography entry title}{fg=}
\setbeamerfont{bibliography entry author}{size=\scriptsize}
\setbeamercolor*{bibliography entry author}{fg=}
\setbeamerfont{bibliography entry location}{size=\scriptsize}
\setbeamercolor*{bibliography entry location}{fg=}
\setbeamerfont{bibliography entry note}{size=\scriptsize}
\setbeamercolor*{bibliography entry note}{fg=}
%    \end{macrocode}
% And kill the abominable icon.
%    \begin{macrocode}
\setbeamertemplate{bibliography item}{\insertbiblabel}
\setbeamerfont{bibliography item}{size=\scriptsize}
\setbeamercolor*{bibliography item}{fg=}
\mode<all>
%    \end{macrocode}
% \iffalse
%</beamerinner>
% \fi
%
% \subsubsection{Presentation outer theme}
% \changes{v1.0}{2019/01/17}{Set the presentation outer theme}
%
% \iffalse
%<*beamerouter>
% \fi
% The outer theme simply sets the header and footer.
%
% \DescribeMacro{\setbeamertemplate{headline text}}
% \DescribeMacro{\setbeamercolor{{headline text}}}
% \DescribeMacro{\setbeamerfont{{headline text}}}
% \DescribeMacro{\setbeamertemplate{{headline}}}
% We want to be able to place a marking in the header without changing
% the overall template.  To do this, we define a text field centered in
% the header.  The font of the text simply inherits from the |headline|.
%    \begin{macrocode}
\setbeamertemplate{headline text}{}
\setbeamercolor{headline text}{parent=headline}
\setbeamerfont{headline text}{parent=headline}
\setbeamertemplate{headline}
{%
    \begin{beamercolorbox}[wd=\textwidth,ht=4em]{headline}
        \null\vfill%
        \hfill{\centering\usebeamertemplate*{headline text}}%
        \hfill\null%
        \vfill\null%
    \end{beamercolorbox}
}
%    \end{macrocode}
%
% \DescribeMacro{\setbeamertemplate{{navigation symbols}}}
% \DescribeMacro{\setbeamertemplate{{page number in head/foot}}}
% \DescribeMacro{\setbeamertemplate{{sidebar right}}}
% We do not want a side bar and we do not want the navigation buttons.
% But, we might want the buttons at some point so we move the navigation
% buttons into the footer and place the slide number on the left margin.
% Using the |page number in head/foot| template allows us to override
% the numbering in the main document; however, we want the default
% option to add frame numbers.
%    \begin{macrocode}
\setbeamertemplate{navigation symbols}{}
\setbeamertemplate{page number in head/foot}[framenumber]%
\setbeamertemplate{sidebar right}{}
%    \end{macrocode}
%
% \DescribeMacro{\defbeamerteplate{{footline text}}}
% \DescribeMacro{\setbeamerfont{{footline text}}}
% \DescribeMacro{\setbeamercolor{{footline text}}}
% Provide a template for placing text in the foot line of the frame.
% The default is to use the short author name and the copy right date in
% the footer.  The font properties are inherited from the
% |page number in head/foot|.
%    \begin{macrocode}
\defbeamertemplate*{footline text}{copyright}{\insertshortauthor{}
                                              \textcopyright{}
                                              \insertshortdate}
\setbeamerfont{footline text}{parent=page number in head/foot}
\setbeamercolor{footline text}{parent=page number in head/foot}
%    \end{macrocode}
%
% \DescribeMacro{\setbeamertemplate{{footline}}}
% And assemble the foot line.  We want the page number and the defined
% footer text.  We place the navigation symbols in the footer, but I
% don't like them so I have them turned off.  Someone could put them
% back if they really wanted.
%    \begin{macrocode}
\setbeamertemplate{footline}{%
    \hspace{0.03\paperwidth}
    \usebeamertemplate*{page number in head/foot}%
    \hfill{\centering\usebeamertemplate*{footline text}}%
    \hfill\usebeamertemplate***{navigation symbols}%
    \vspace{0.02\paperheight}%
}
\mode<all>
%    \end{macrocode}
% \iffalse
%</beamerouter>
% \fi
%
% \Finale{}
\endinput

%--------1---------2---------3---------4---------5---------6---------7--
% This is collection of stuff that I find very useful.
%
% This version is used for collecting my notation for my thesis.
%--------1---------2---------3---------4---------5---------6---------7--
% First we include all of our desired packages
\usepackage{amsmath}
\usepackage{amsfonts}
\usepackage{graphicx}
\usepackage{bm}

%--------1---------2---------3---------4---------5---------6---------7--
% Use a R 3 vector and dyad similar to the APS style.
\newcommand{\svec}[1]{
    \ensuremath{ \bm{#1} }
}
\newcommand{\dyad}[1]{
    \ensuremath{ \overleftrightarrow{\svec{#1}} }
}

%--------1---------2---------3---------4---------5---------6---------7--
% We define generic unit vectors and a few specific
\newcommand{\unitvec}[1]{
    \ensuremath{ \bm{ \hat{#1} } }
}
% Generic
\newcommand{\ihat}{\unitvec{\imath}}
\newcommand{\jhat}{\unitvec{\jmath}}
\newcommand{\khat}{\unitvec{k}}
% Cartesian
\newcommand{\xhat}{\unitvec{x}}
\newcommand{\yhat}{\unitvec{y}}
\newcommand{\zhat}{\unitvec{z}}
% Spherical
\newcommand{\rhat}{\unitvec{r}}
\newcommand{\thetahat}{\unitvec{\theta}}
\newcommand{\phihat}{\unitvec{\phi}}
% Cylindrical
\newcommand{\rhohat}{\unitvec{\rho}}

%--------1---------2---------3---------4---------5---------6---------7--
% Next a few quantum mechanics commands
\newcommand{\bra}[1]{
    \ensuremath{ \bigl\langle#1\bigr\rvert }
}
\newcommand{\ket}[1]{
    \ensuremath{ \bigl\lvert#1\bigr\rangle }
}
\newcommand{\braket}[2]{
    \ensuremath{ \bigl\langle#1\big\vert#2\bigr\rangle }
}
\newcommand{\operator}[1]{
    \ensuremath{ \mathbb{#1} }
}

%--------1---------2---------3---------4---------5---------6---------7--
% Now for some functions
\let\divsymb=\div % Redefine so we can use \div
\newcommand{\abs}[1]{
    \ensuremath{ \left\lvert#1\right\rvert }
}
\newcommand{\norm}[1]{
    \ensuremath{ \left\lVert#1\right\rVert }
}
\newcommand{\grad}[1][]{
    \ensuremath{ \bm{\nabla}_{#1} }
}
\renewcommand{\div}[1][]{
    \ensuremath{ \bm{\nabla}_{#1}\cdot }
}
\newcommand{\curl}[1][]{
    \ensuremath{ \bm{\nabla}_{#1}\times }
}

%--------1---------2---------3---------4---------5---------6---------7--
% And change varepsilon to something more reasonable
\let\eps=\varepsilon

%--------1---------2---------3---------4---------5---------6---------7--
% A quick way to differentiate a mathematical vector and matrix
% from a R_3 Cartesian vector and dyad
\newcommand{\mat}[1]{
    \ensuremath{ \overline{\overline{\bm{#1}}} } 
}
\newcommand{\mvec}[1]{
    \ensuremath{ \overline{\bm{#1}} } 
}

%--------1---------2---------3---------4---------5---------6---------7--
%
